\documentclass[a4paper]{article}
\usepackage[14pt]{extsizes}
\usepackage[utf8]{inputenc}
\usepackage[russian]{babel}
\usepackage{setspace,amsmath}
\usepackage{fancyvrb}
\usepackage{graphicx}
\graphicspath{{pictures/}}
\DeclareGraphicsExtensions{.png,.jpg}
%\DefineShortVerb{\|}
%\usepackage[argument]{graphicx}
%\DeclareGraphicsExtensions{.pdf,.png,.jpg}
\usepackage[colorlinks,urlcolor=blue]{hyperref}
\usepackage[left=20mm, top=15mm, right=15mm, bottom=15mm, nohead, footskip=10mm]{geometry} % настройки полей документа



\begin{document}

\begin{titlepage}
	\begin{center}
		МИНИСТЕРСТВО ОБРАЗОВАНИЯ И НАУКИ РОССИЙСКОЙ ФЕДЕРАЦИИ
		ФЕДЕРАЛЬНОЕ ГОСУДАРСТВЕННОЕ АВТОНОМНОЕ ОБРАЗОВАТЕЛЬНОЕ УЧРЕЖДЕНИЕ
		ВЫСШЕГО ОБРАЗОВАНИЯ
		«САНКТ-ПЕТЕРБУРГСКИЙ ГОСУДАРСТВЕННЫЙ УНИВЕРСИТЕТ
		АЭРОКОСМИЧЕСКОГО ПРИБОРОСТРОЕНИЯ» \\
		\vspace{1cm}
		КАФЕДРА КОМПЬЮТЕРНЫХ ТЕХНОЛОГИЙ И ПРОГРАММНОЙ ИНЖЕНЕРИИ
	\end{center}

	\vspace{1cm}
	\begin{flushleft}
		КУРСОВОЙ ПРОЕКТ \\
		ЗАЩИЩЕН С ОЦЕНКОЙ \\
		РУКОВОДИТЕЛЬ \\
	\end{flushleft}

	\begin{tabular}{p{4cm} p{0.5cm} p{4cm} p{0.5cm} p{4cm}}
		\centering ст.преп. & & & & \hspace{0.9cm} Поляк М.Д. \\
		\cline{1-1} \cline{3-3} \cline{5-5}
		\centering \tiny{должность, уч. степень, звание} & &
		\centering \tiny{подпись, дата} & &
		\centering \tiny{инициалы, фамилия}
	\end{tabular}

	\begin{center}
		\begin{tabular}{p{13cm}}
			\vspace{2.5cm} \\
			\begin{center}
				ПОЯСНИТЕЛЬНАЯ ЗАПИСКА \\
				К КУРСОВОМУ ПРОЕКТУ \\
				\vspace{0.5cm}
				СОЗДАНИЕ USB ДРАЙВЕРА \\
				\vspace{0.5cm}
				по дисциплине: ОПЕРАЦИОННЫЕ СИСТЕМЫ И СЕТИ
			\end{center}
		\end{tabular}
	\end{center}

	\vspace{2.5cm}
	\begin{flushleft}
		РАБОТУ ВЫПОЛНИЛ
	\end{flushleft}

	\begin{tabular}{p{3cm} p{1cm} p{0.5cm} p{3.5cm} p{0.5cm} p{3.5cm}}
		СТУДЕНТ ГР. & 4336  & & & & Синяев А.И. {} \\
		\cline{2-2} \cline{4-4} \cline{6-6}
		& & & \centering \tiny{подпись, дата}
		& & \centering \tiny{инициалы, фамилия}
	\end{tabular}

	\begin{center}
		\vspace{1cm}
		Санкт-Петербург \\
		2016
	\end{center}
\end{titlepage}

\newpage

\section{Цель работы}

\normalsize Цель работы: Знакомство с устройством ядра ОС Linux. Получение опыта разработки драйвера устройства.

\section{Задание(4 вариант)}

Добавление защиты от несанкционированного запуска операционной системы.
Необходимо внести изменения в процесс загрузки ядра Linux, добавив проверку наличия
подключенного через интерфейс USB flash-накопителя с заданным серийным номером.
Если в процессе загрузки операционной системы нужный flash-накопитель подключен к
одному из портов USB, то операционная система успешно загружается в штатном режиме.
Если flash-накопитель с нужным серийным номером отсутствует, необходимо
приостановить загрузку операционной системы и предоставить пользователю три попытки
подключить нужный flash-накопитель. Если пользователь три раза подключает flash-
накопитель с неправильным серийным номером, произвести выключение компьютера.
Если же пользователь подключит нужный flash-накопитель, продолжить загрузку
операционной системы в штатном режиме.

\section{Техническая документация}
\begin{enumerate}
	\item Сборка проекта: \\
		Скачиваем файлы с репозитория на github при помощи команды: \begin{verbatim}
	git clone https://github.com/3a9LL/usb_auth.git
\end{verbatim} 
	\item Сборка и добавление в автозагрузку: \\
	Шаг 1: Собираем драйвер (test.ko) с помощью запуска команды "make".
	 \\
	Шаг 2: Выставляем права на выполнение файла 1.sh командой "chmod +x 1.sh".
	  \\
	Шаг 3: Запускаем скрипт 1.sh командой "sudo ./1.sh".
	  \\
	Шаг 4: Отключаем флеш-устройство при загрузке системы.
	 \\
	Шаг 5: Перезгружаем систему.
	  \\
	Шаг 6:	Все работает, система требует флешку для запуска.
\end{enumerate}
\newpage

\section{Скриншоты}
\begin{figure}[htbp]
  \centering
  \includegraphics[width=0.8\textwidth]{1}
  \caption{Вставка неправильной флешки дважды c последующей вставкой корректной флешки}\label{fig:1}
\end{figure}  
\begin{figure}[htbp]
  \centering
  \includegraphics[width=0.8\textwidth]{2}
  \caption{Вставка неправильной флешки третий раз перезагрузка компьютера}\label{fig:1}
\end{figure}

\section{Выводы}
	В процессе выполнения данной курсовой работы мною были получены знания и навыки, необходимые для работы с ядром ОС Linux, а так же знания и навыки в разработке драйверов устройств.
\newpage
\section{Приложение}
\begin{verbatim}
test.c:

#include <linux/kernel.h>
#include <linux/module.h>
#include <linux/usb.h>
#include <linux/sched.h>
#include <linux/kthread.h>
#include <linux/types.h>
#include <linux/tty.h>    
#include <linux/version.h> 
#include <linux/delay.h> 
#include <linux/reboot.h> 

struct task_struct *tAgetty;
struct task_struct *task;
bool stopThread = true;
bool isTry = true;
static int param = 1;
module_param( param, int, 0 );
int countTry = 3;

static int thread_agetty_uninterrupyible( void * data) 
{
	// основной цикл потока
	while(stopThread)
	{
		for_each_process(task)
		{

			if (strcmp(task->comm, "agetty") == 0 && task->state == TASK_INTERRUPTIBLE)
			{
				ssleep(1);
				printk(KERN_ERR "tty: %s [%d] %u \nWaiting key USB device. %d attempts\n", task->comm , task->pid, (u32)task->state, countTry);
				task->state = TASK_UNINTERRUPTIBLE;
			}
		}
	}
	if (countTry <= 0)
	{
		printk(KERN_ERR "Reboot in 3...\n");
		ssleep(1);
		printk(KERN_ERR "Reboot in 2...\n");
		ssleep(1);
		printk(KERN_ERR "Reboot in 1...\n");
		ssleep(1);
		kernel_restart(NULL);		
	}

	return -1;
}

static int pen_probe(struct usb_interface *interface, const struct usb_device_id *id)
{
	struct usb_device *dev = interface_to_usbdev(interface);

	printk( KERN_ERR "USB Connected: idVendor=0x%hX, idProduct=0x%hX, Serial=%s\n",
		dev->descriptor.idVendor,
		dev->descriptor.idProduct, dev->serial ); 
	
	if (isTry)
	{
		if (strcmp(dev->serial,"173711115C60F42B")==0)
		{
			stopThread = false;
			isTry = false;
			ssleep(1);
			printk( KERN_ERR "Key USB device connected\n");

			for_each_process(task)
			{
				if (strcmp(task->comm, "agetty") == 0 && task->state == TASK_UNINTERRUPTIBLE)
				{
					//printk(KERN_ERR "flash: %s [%d] %u \n", task->comm , task->pid, (u32)task->state);
					task->state = TASK_INTERRUPTIBLE;
				}
			}
		} else {

			countTry--;
			printk(KERN_ERR "Tries left:  %i \n", countTry);
			if (countTry <= 0)
			{
				stopThread = false;
			}
		}		
	}

	
	return 0;
}

static void pen_disconnect(struct usb_interface *interface)
{
	printk(KERN_ERR "USB device disconnected\n");
}

static struct usb_device_id pen_table[] =
{
	{ .driver_info = 42 },
    	{}
};

static struct usb_driver pen_driver =
{
	.name = "usb_auth",
	.probe = pen_probe,
	.disconnect = pen_disconnect,
	.id_table = pen_table,
};

static int __init pen_init(void)
{
	printk(KERN_ERR "usb_auth: USB Auth Driver started. 3 attempts\n");

	// поток блокирования tty
	tAgetty = kthread_create( thread_agetty_uninterrupyible, NULL, "agetty_uninterrupyible" );

	if (!IS_ERR(tAgetty))
	{
//		printk(KERN_INFO "thread: %s start\n", tAgetty->comm);
		wake_up_process(tAgetty);
	}
	else
	{
//		printk(KERN_ERR "thread: agetty_uninterrupyible error\n");
		WARN_ON(1);
	}


	return usb_register(&pen_driver);
}

static void __exit pen_exit(void)
{
	usb_deregister(&pen_driver);
}

module_init(pen_init);
module_exit(pen_exit);

MODULE_LICENSE("GPL");
MODULE_AUTHOR("none");
MODULE_DESCRIPTION("USB Auth Driver");

1.sh:

#!/bin/bash
rm -f /usr/lib/modules/$(uname -r)/test.ko
cp test.ko /usr/lib/modules/$(uname -r)/
echo 'test' > /etc/modules-load.d/test.conf
depmod
\end{verbatim}
\end{document}  
